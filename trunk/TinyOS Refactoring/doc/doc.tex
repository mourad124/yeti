\documentclass[a4paper,10pt]{report}
\usepackage[utf8x]{inputenc}
\usepackage{listings} 
\lstset{numbers=left, numberstyle=\tiny, numbersep=5pt} 
\lstset{language=Java} 


% Title Page
\title{Refactoring Extenstion for the Yeti NesC Eclipse Plugin}
\author{Noah Heusser, Max Urech}


\begin{document}
\maketitle

\begin{abstract}
\end{abstract}

\chapter{What is refactoring}

\chapter{Eclipse Plugins}
\section{General Infos about writeing Plugins}
Eclipse is famous for it's plugin architecture. Everything in Eclipse is a Plugin. A Plugin is a JAR-File or a Folder in the plugins-Directory of the eclipse program folder.
To be a plugin it takes at least three Files.
\begin{itemize}
  \item META-INF/MANIFEST.MF
  \item plugin.xml
  \item plugin.class
\end{itemize}

\subsection{META-INF/MANIFEST.MF}
The Manifest file is the first File that is read by Eclipse while loding the Plugin. It conatins all the Informations about what requirements 
are needed to load the Plugin and how it can be loaded. I will now explain the most important entrys in the Manifest file.

\begin{lstlisting}[caption=MANIFEST]{META-INF/MANIFEST.MF}
Manifest-Version: 1.0
Bundle-ManifestVersion: 2
Bundle-Name: TinyOS_Refactoring
Bundle-SymbolicName: tinyos.yeti.refactoring;singleton:=true
Bundle-Version: 1.0.0.qualifier
Bundle-Activator: tinyos.yeti.refactoring.RefactoringPlugin
Require-Bundle: org.eclipse.ui,
 org.eclipse.core.runtime,
 tinyos.yeti.core;bundle-version="2.2.17",
 tinyos.yeti.parser.nesc12;bundle-version="1.2.17",
 org.eclipse.ui.workbench.texteditor;bundle-version="3.5.1",
 org.eclipse.ltk.core.refactoring;bundle-version="3.5.0",
 org.eclipse.ltk.ui.refactoring;bundle-version="3.4.101",
 org.eclipse.core.resources;bundle-version="3.5.2",
 org.eclipse.jface.text;bundle-version="3.5.2",
 org.eclipse.ui.ide;bundle-version="3.5.2",
 org.eclipse.ui.editors;bundle-version="3.5.0",
 tinyos.yeti.preprocessor.nesc12;bundle-version="1.2.17",
 org.eclipse.core.expressions;bundle-version="3.4.101"
Bundle-ActivationPolicy: lazy
Bundle-RequiredExecutionEnvironment: JavaSE-1.6
\end{lstlisting}

The first two lines define that this is a Manifest for a OSGi Bundle. OSGi is used by Eclipse for organizing it's Plugins.
th Bundle-Name is an Internaly used name for the OSGi Framework. 

\begin{description}
 \item[Bundle-Activator] This is the Path of the Class which is used to start the whole Plugin. The class must extend org.eclipse.core.runtime.Plugin. 
 \item[Require-Bundle]
 \end{description}


\section{Language Toolkit for Processor Based Refactoring}
For our refactoring Plugin we used the Processor based refactoring, offered by the Language Toolkit of Eclipse. 

\section{Menu's with conditional visibility}

3x new method:
getInternalyUnusedDeclarations
isModifying(Identifier) checks if the Identifier is modified in the Codeblock
getPotentionalyChangedVariables

\end{document}          
